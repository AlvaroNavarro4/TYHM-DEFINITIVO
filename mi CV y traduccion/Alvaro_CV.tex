%%%%%%%%%%%%%%%%%
% This is an example CV created using altacv.cls (v1.3, 10 May 2020) written by
% LianTze Lim (liantze@gmail.com), based on the
% Cv created by BusinessInsider at http://www.businessinsider.my/a-sample-resume-for-marissa-mayer-2016-7/?r=US&IR=T
%
%% It may be distributed and/or modified under the
%% conditions of the LaTeX Project Public License, either version 1.3
%% of this license or (at your option) any later version.
%% The latest version of this license is in
%%    http://www.latex-project.org/lppl.txt
%% and version 1.3 or later is part of all distributions of LaTeX
%% version 2003/12/01 or later.
%%%%%%%%%%%%%%%%

%% If you are using \orcid or academicons
%% icons, make sure you have the academicons
%% option here, and compile with XeLaTeX
%% or LuaLaTeX.
% \documentclass[10pt,a4paper,academicons]{altacv}

%% Use the "normalphoto" option if you want a normal photo instead of cropped to a circle
% \documentclass[10pt,a4paper,normalphoto]{altacv}

\documentclass[10pt,a4paper,ragged2e,withhyper]{altacv}

%% AltaCV uses the fontawesome5 and academicon fonts
%% and packages.
%% See http://texdoc.net/pkg/fontawesome5 and http://texdoc.net/pkg/academicons for full list of symbols. You MUST compile with XeLaTeX or LuaLaTeX if you want to use academicons.

% Change the page layout if you need to
\geometry{left=1.25cm,right=1.25cm,top=1.5cm,bottom=1.5cm,columnsep=1.2cm}

% The paracol package lets you typeset columns of text in parallel
\usepackage{paracol}


% Change the font if you want to, depending on whether
% you're using pdflatex or xelatex/lualatex
\ifxetexorluatex
  % If using xelatex or lualatex:
  \setmainfont{Lato}
\else
  % If using pdflatex:
  \usepackage[default]{lato}
\fi

% Change the colours if you want to
\definecolor{VividPurple}{HTML}{3E0097}
\definecolor{SlateGrey}{HTML}{2E2E2E}
\definecolor{LightGrey}{HTML}{666666}
% \colorlet{name}{black}
\colorlet{tagline}{VividPurple}
\colorlet{heading}{VividPurple}
\colorlet{headingrule}{VividPurple}
% \colorlet{subheading}{PastelRed}
\colorlet{accent}{VividPurple}
\colorlet{emphasis}{SlateGrey}
\colorlet{body}{LightGrey}

% Change some fonts, if necessary
% \renewcommand{\namefont}{\Huge\rmfamily\bfseries}
% \renewcommand{\personalinfofont}{\footnotesize}
% \renewcommand{\cvsectionfont}{\LARGE\rmfamily\bfseries}
% \renewcommand{\cvsubsectionfont}{\large\bfseries}

% Change the bullets for itemize and rating marker
% for \cvskill if you want to
\renewcommand{\itemmarker}{{\small\textbullet}}
\renewcommand{\ratingmarker}{\faCircle}

%% sample.bib contains your publications
\addbibresource{sample.bib}

\begin{document}
\name{Navarro Cangas Alvaro}
\tagline{Estudiante avanzado de Ingenieria Industrial - Innovador y Optimizador }
% Cropped to square from https://en.wikipedia.org/wiki/Marissa_Mayer#/media/File:Marissa_Mayer_May_2014_(cropped).jpg, CC-BY 2.0
%% You can add multiple photos on the left or right
\photoR{4cm}{alv.jpeg}
% \photoL{2cm}{Yacht_High,Suitcase_High}
\personalinfo{%
  % Not all of these are required!
  % You can add your own with \printinfo{symbol}{detail}
  \email{alvaronavarro444@gmail.com}
   \phone{+54 2622 (15)658643}
  \mailaddress{Guemes 1222 Tunuyan}
  \location{Mendoza, Argentina}
  \homepage{https://es-la.facebook.com/alvaronavarro44}
  \twitter{@alvaronavarro_444}
  \linkedin{alvaro-navarro-cangas-17388820a/}
%   \github{github.com/mmayer} % I'm just making this up though.
%   \orcid{orcid.org/0000-0000-0000-0000} % Obviously making this up too. If you want to use this field (and also other academicons symbols), add "academicons" option to \documentclass{altacv}
  %% You MUST add the academicons option to \documentclass, then compile with LuaLaTeX or XeLaTeX, if you want to use \orcid or other academicons commands.
  % \orcid{0000-0000-0000-0000}
  %% You can add your own arbtrary detail with
  %% \printinfo{symbol}{detail}[optional hyperlink prefix]
  % \printinfo{\faPaw}{Hey ho!}
  %% Or you can declare your own field with
  %% \NewInfoFiled{fieldname}{symbol}[optional hyperlink prefix] and use it:
  % \NewInfoField{gitlab}{\faGitlab}[https://gitlab.com/]
  % \gitlab{your_id}
}

\makecvheader

%% Depending on your tastes, you may want to make fonts of itemize environments slightly smaller
\AtBeginEnvironment{itemize}{\small}

%% Set the left/right column width ratio to 6:4.
\columnratio{0.6}

% Start a 2-column paracol. Both the left and right columns will automatically
% break across pages if things get too long.
\begin{paracol}{2}

\cvsection{Desempeño académico}

\cvevent{Conocimientos y experiencias}{Mar. 2018  Abril 2021}{Mendoza Argentina}{Facultad de Ingeniería UN CUYO }

\begin{itemize}
\item Formación en ciencias básicas y duras
\item Gestión en división de tareas
\item Uso de simulaciones en Proteus
\item Capacitación en uso de WinLog
\item Organización y formulación de proyectos
\item Uso de programas para la resolución de modelos matemáticos
\item Uso de OCTAVE para la resolución y gráfica de problemas

\end{itemize}

\divider

\cvevent{Secretaria de Ciencias Básicas UN CUYO}{Ayudante de cátedra de Análisis Matemático II}{2018-2019}{Mendoza, Argentina}
\begin{itemize}
\item Preparación de documentación y material bibliográfico para cátedra de Análisis Matemático II
\item Generación de Trabajos Prácticos y ayuda personalizada en horarios acordados con los estudiantes que lo soliciten
\end{itemize}

\divider

\cvevent{Capacitación en Cursos de Office} {Cursos de Excel y Word}{2019 -2020}{Instituto de computación}

\begin{itemize}
\item Gráficas de dispersión
\item Uso de distribuciones de probabilidad
\item Generación de tendencias según datos
\item Conocimientos en configuración de textos e hipertextos 
\end{itemize}

\divider

\cvsection{Experiencias extracurriculares}
\cvevent{Participaciones deportivas}{Mar. 2015  Abril 2019}{Buenos Aires Argentina}{CENARD}

\begin{itemize}
\item Participación en competencias en representación del país
\item Participación en torneos nacionales representando la provincia
\item Participación en proceso de selección para Juegos Olímpicos
\item Conocimiento de lugares turísticos y centros de desarrollo deportivo 
\item Entrevistas en base a la experiencia de campeonatos

\end{itemize}


% \divider

% \cvevent{Product Engineer}{Google}{23 June 1999 -- 2001}{Palo Alto, CA}

% \begin{itemize}
% \item Joined the company as employe \#20 and female employee \#1
% \item Developed targeted advertisement in order to use user's search queries and show them related ads
% \end{itemize}

\cvsection{Mi día de trabajo}

% Adapted from @Jake's answer from http://tex.stackexchange.com/a/82729/226
% \wheelchart{outer radius}{inner radius}{
% comma-separated list of value/text width/color/detail}
% Some ad-hoc tweaking to adjust the labels so that they don't overlap
\hspace*{-1em}  %% quick hack to move the wheelchart a bit left
\wheelchart{1.5cm}{0.5cm}{%
  10/13em/accent!30/Descanso para crear mis sueños,
  25/9em/accent!60/Tiempo de cursado,
  5/11em/accent!10/\footnotesize\\[Actividades recreativas],
  20/11em/accent!40/Familia y amigos,
  5/8em/accent!20/\footnotesize Tiempo dedicado a redes sociales y mi blog,
  30/9em/accent/Estudio\ \mbox{.} Afianzo conocimientos,
  5/8em/accent!20/Entrenamiento y GYM
}

% use ONLY \newpage if you want to force a page break for
% ONLY the currentc column
\newpage

\switchcolumn

\cvsection{Filosofía de Vida}
\begin{quote}
``Cada uno forma su camino y vive a su manera''
\end{quote}

\cvsection{Estoy Orgulloso de:}

\cvachievement{\faTrophy}{Haber participado en torneos representando a mi provincia y a mi país }

\divider

\cvachievement{\faHeartbeat}{Aprender en base a experiencias en el deporte diversas formas de desenvolverse en situaciones de la vida que nos llevan a formarnos como personas}

\divider

\cvachievement{\faChartLine}{Poder estudiar Ingeniería}{El entrar a esta carrera ayuda a vincularme con estudiantes y profesores realmente dedicados a su labor lo que ayuda a perfeccionarme como persona y optimizar mi desarrollo academico}


\cvsection{Fortalezas}

\cvtag{Responsable}
\cvtag{Persuasivo y Tenaz en el trabajo}\\
\cvtag{Con iniciativa propia}
\cvtag{Persistente y Consistente}
\cvtag{Objetivista}
\divider\smallskip

\cvsection{Idiomas}

\cvskill{Inglés}{5}
% \divider

\cvskill{Francés}{2}
% \divider



\cvsection{Educación}

\cvevent{Finalización de secundario con especialización en Economía y Administración}{2018}{Colegio Niño Jesús}{Tunuyan Mendoza}

\divider

\cvevent{Ingeniero Industrial 21 materias aprobadas }{Universidad Nacional de Cuyo}{2018 2021}{}

\newpage

\cvsection{Referees}

% \cvref{name}{email}{mailing address}
\cvref{Prof.\ Ricardo Palma}{Instituto}{Ingeniería Industrial}
{Universidad Nacional de Cuyo}

\divider

\cvref{Prof.\ Gustavo Masera}{Institute}{EPISTEME}
{CEAL- Centro de Estudios y Apicaciones Logísticas}

\end{paracol}

\end{document}
